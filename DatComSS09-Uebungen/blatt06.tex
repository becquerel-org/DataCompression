\documentclass[a4paper]{article}

\usepackage[german]{babel}
%\usepackage{umlaut}
\usepackage{graphicx}
\usepackage{amsfonts}
\usepackage{fullpage}
\usepackage{hyperref}
\usepackage[latin1]{inputenc}

\def\header#1#2#3#4{\pagestyle{empty}
\noindent
\begin{minipage}[t]{0.6\textwidth}
\begin{flushleft}
\bf \"Ubungen zur Datenkompression\\
WSI f\"ur Informatik\\
Lange/Krebs (Behle)
\end{flushleft}
\end{minipage}
\begin{minipage}[t]{0.4\textwidth}
\begin{flushright}
\bf Sommersemester 2009\\
Universit\"at T\"ubingen\\
#2 %Datum eintragen
\end{flushright}
\end{minipage}

\begin{center}
{\Large\bf Blatt #1}

{(Abgabe am #3)}
\end{center}
}

\begin{document}
\header{6}{25.06.2009}{02.07.2009}{}

\bigskip

\noindent
Das PNG-Format beschreibt f"unf Vorfiltertypen, die auf die Pixel eines Bildes angewendet werden k"onnen, bevor das Bild komprimiert wird.

Siehe auch folgender Link: \url{http://www.gnupdf.org/PNG_and_TIFF_Predictors_Filter}
\begin{description}
\item[None] Es findet keine Vorfilterung statt.
\item[Sub] Es werden die Differenzen zum jeweils linken Pixel betrachtet.
\item[Up]  Es werden die Differenzen zum jeweils dar"uberliegenden Pixel betrachtet.
\item[Average] Es wird die Differenz zum Mittelwert des linken und dar"uberliegenden Pixels betrachtet.
\item[Paeth] Es wird die Differenz zum Paeth-Pr"adiktor betrachtet, der aus dem linken, oberen und schr"ag links oben liegenden Pixel berechnet wird, siehe Link.
\end{description}


\bigskip


{\bf Aufgabe 1  \quad(Schwarzweissbilder)}\\
%\begin{enumerate}
%\item
Wenden Sie auf das Bild des letzten "Ubungsblattes die ersten 3 Filtertypen and und vergleichen Sie die Entropie der Ausgaben.

Die Differen wird mit XOR ausgerechnet.
%\item "Uberlegen Sie sich, einen Pr"adiktor, der aus dem linken, oberen und schr"ag links oben liegenden Pixel, eine Sch"atzwert ermittelt.
%\end{enumerate}


\bigskip


{\bf Aufgabe 2  \quad(Farbbilder)}\\
Laden Sie von der "Ubungsseite das Farbbild herunter.
Wenden Sie alle 5 Filtertypen auf das Farbbild an und vergleichen Sie die Entropie. 

Denken Sie daran, die Differenzen modulo 256 zu berechnen.

Dazu k"onnen Sie die Klasse {\rm PNGReader} von Blatt 5 benutzen, wenn Sie sie folgenderma{\ss}en adaptieren:
Die {\rm getPixel}-Funktion gibt jetzt nicht mehr einen {\rm int}, sondern einen Array von 3 Integern zur"uck.
Die Werte liegen jedoch im Bereich $[0,255]$. 


\bigskip


{\bf Aufgabe 3  \quad(Relation zischen den Farbkan"alen)\quad (Nur Diplom und Master)}\\
Berechnen Sie f"ur das Bild aus Aufgabe zwei die bedingten Entropien zwischen je zwei Farbkan"alen und die Entropien der drei Farbkan"ale.


\end{document}

