\documentclass[a4paper]{article}

\usepackage[german]{babel}
%\usepackage{umlaut}
\usepackage{graphicx}
\usepackage{amsfonts}
\usepackage{fullpage}
\usepackage[latin1]{inputenc}

\def\header#1#2#3#4{\pagestyle{empty}
\noindent
\begin{minipage}[t]{0.6\textwidth}
\begin{flushleft}
\bf \"Ubungen zur Datenkompression\\
WSI f\"ur Informatik\\
Lange/Krebs (Behle)
\end{flushleft}
\end{minipage}
\begin{minipage}[t]{0.4\textwidth}
\begin{flushright}
\bf Sommersemester 2009\\
Universit\"at T\"ubingen\\
#2 %Datum eintragen
\end{flushright}
\end{minipage}

\begin{center}
{\Large\bf Blatt #1}

{(Abgabe am #3)}
\end{center}
}

\begin{document}
\header{4}{28.05.2009}{11.06.2009}{}

\bigskip
Bachelorstudenten m"ussen nur die Aufgaben 2 und 3 bearbeiten.

\bigskip

{\bf Aufgabe 1  \quad(Entropie)}\\
Zeigen Sie: Ist $X$ eine echt unabh"angige Zufallsquelle ohne Ged"achtnis (die Wahrscheinlichkeit f"ur ein Zeichen ist unabh"angig von den vorangegangenen Zeichen), so gilt:
$$2\cdot H(X)= H(X^2).$$
Dabei ist $X^2$, die Zufallsquelle, welche entsteht, wenn man statt einem Zeichen immer zwei aufeinanderfolgende Zeichen betrachtet.

\bigskip

{\bf Aufgabe 2  \quad(Verh"altnis Signal zu Verzerrung)}\\
Schreiben Sie Routinen, die aus zwei Folgen von {\rm double}-Werten folgende Werte berechnen: Mittlerer quadratischer Fehler, Signal to Noise, Signal to Noise in Dezibel.

Die erhalten die Eingabefolgen in Form von {\rm DataInput}-Objekten. Nutzen Sie die {\rm readDouble}-Methode um die Eingaben zu lesen.


\bigskip

{\bf Aufgabe 3  \quad(LZW)}\\
Das Anfangw"orterbuch einer Folge, die mit dem LZW-Algorithmus kodiert wurde, sieht wie folgt aus:
\begin{center}
\begin{tabular}{|l|l|}\hline
Index & Eintrag\\\hline
1 & a\\
2 & b\\
3 & r\\
4 & t\\\hline
\end{tabular}
\end{center}
\begin{enumerate}
\item Dekodieren Sie die nachstehende Ausgabe des LZW-Kodierers:
$$3,1,4,6,8,4,2,1,2,5,10,6,11,13,6$$
\item Testen Sie Ihr Ergebnis, indem Sie mit Hilfe des obigen Anfangsw"orterbuchs die von Ihnen dekodierte Folge wieder kodieren.
\end{enumerate}

\end{document}

