\documentclass[a4paper]{article}

\usepackage[german]{babel}
%\usepackage{umlaut}
\usepackage{graphicx}
\usepackage{amsfonts}
\usepackage[latin1]{inputenc}

\def\header#1#2#3#4{\pagestyle{empty}
\noindent
\begin{minipage}[t]{0.6\textwidth}
\begin{flushleft}
\bf \"Ubungen zur Datenkompression\\
WSI f\"ur Informatik\\
Lange/Krebs (Behle)
\end{flushleft}
\end{minipage}
\begin{minipage}[t]{0.4\textwidth}
\begin{flushright}
\bf Sommersemester 2009\\
Universit\"at T\"ubingen\\
#2 %Datum eintragen
\end{flushright}
\end{minipage}

\begin{center}
{\Large\bf Blatt #1}

{(Abgabe am #3)}
\end{center}
}

\begin{document}
\header{1}{30.04.2009}{5.5.2009}{}

Beachten Sie bitte die Hinweise zu den "Ubungen auf der R"uckseite!

\bigskip

{\bf Aufgabe 1  \quad(Entropie)}\\
Eine Informationsquelle sende Buchstaben aus dem Alphabet $\Sigma=\{a,b,c,d,e\}$ mit den Wahrscheinlichkeiten $P(a)=0,15$, $P(b)=0,04$, $P(c)=0,26$, $P(d)=0,05$ und $P(e)=0,5$.
\begin{enumerate}
\item Berechnen Sie die Entropie der Quelle!
\item Bestimmen Sie einen Huffmann-Code f"ur diese Quelle per Hand!
\end{enumerate}

\bigskip

{\bf Aufgabe 2  \quad(Programmieraufgaben)}\\
Bitte beachten Sie die Hinweise zu den Programmieraufgaben.

\smallskip
Schreiben Sie Routinen, die folgende Aufgaben erf"ullen:
\begin{enumerate}
\item Einfache Statistik: Gibt an, welche Bytes in einer Datei vorkommen und wie oft jedes Byte vorkommt.
\item Entropie: Eingabe eine Liste von Zeichenh"aufigkeiten einer Quelle, Ausgabe die Entropie der Quelle.
\item Entropie: Berechnet die Entropie einer Datei, wobei als Zeichen Bytes verwendet werden.
\item(*) Erweitern Sie obige Routinen, so dass sie auch mit Alphabete von $n$ Bytes arbeiteten. Hat die Eingabe nicht eine L"ange, die ein vielfaches von $n$ ist, so padden sie sie mit 0-Bytes.
\end{enumerate}


\bigskip

{\bf Aufgabe 3  \quad(Empirische Tests)}\\
\begin{enumerate}
\item(*) Generieren Sie eine Datei, bei der der Kompressionsquotient gr"o{\ss}er als 1 ist!
\item\label{a} Laden sie die Beispieldateien zum "Ubungsblatt herunter und nutzen Sie Ihr Programm um die Entropien der Dateien byteweise zu ermitteln!
\item Berechnen Sie mit Hilfe der Entropie die erwarteten minimalen Codel"angen der \verb$*.tex$-Dateien. Wenden sie ein Programm wie \verb$gzip$ auf diese Dateien an und Vergleichen sie deren Gr"o{\ss}e mit der erwarteten minimalen Codel"ange.

Interpretieren Sie das Ergebnis!
\item(*) Ermitteln Sie die Entropien der Dateien aus Teil \ref{a} f"ur Alphabetgr"o{\ss}en \`a 3 und 3 Byte!
\end{enumerate}

\bigskip 
{\bf Link zu den \"Ubungen:}\\
\verb$http://www-fs.informatik.uni-tuebingen.de/lehre/ss09/datkom/uebungen.shtml$

\newpage

{\bf Programmieraufgaben}\\
Die f"ur die Programmieraufgaben vorgesehene Programmiersprache ist Java. Wenn Sie die "Ubungen in einer anderen Programmiersprache abgeben wollen, so wenden Sie sich an den Betreuer der "Ubungsgruppen.

Allgemein: Die Abgaben erfolgen elektronisch in Form von gut dokumentiertem und compilierf"ahigen Quellcode.

Sofern nicht anders angegeben, sollen die geforderten Programme die zu bearbeitende Datei als ersten Kommandozeilenparameter erhalten.

Die auf diesem Blatt erstellten Routinen werden in den weiteren "Ubungen gebraucht werden. Erstellen Sie die in Aufgabe 2 geforderten Aufgaben in Biblotheksform und erstellen Sie mit deren Hilfe die f"ur Aufgabe 3 geforderten Programme.

Nutzen Sie f"ur das Einlesen der Datei in Java die Klasse \verb$FileInputStream$, um die Datei in Form von Bytes zu einzulesen.

Unter Windows k"onnen die Ergebnisse wegen der unterschiedlichen Darstellung des Zeilenendeymbols abweichen.
\end{document}


