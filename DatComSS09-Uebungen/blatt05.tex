\documentclass[a4paper]{article}

\usepackage[german]{babel}
%\usepackage{umlaut}
\usepackage{graphicx}
\usepackage{amsfonts}
\usepackage{fullpage}
\usepackage[latin1]{inputenc}

\def\header#1#2#3#4{\pagestyle{empty}
\noindent
\begin{minipage}[t]{0.6\textwidth}
\begin{flushleft}
\bf \"Ubungen zur Datenkompression\\
WSI f\"ur Informatik\\
Lange/Krebs (Behle)
\end{flushleft}
\end{minipage}
\begin{minipage}[t]{0.4\textwidth}
\begin{flushright}
\bf Sommersemester 2009\\
Universit\"at T\"ubingen\\
#2 %Datum eintragen
\end{flushright}
\end{minipage}

\begin{center}
{\Large\bf Blatt #1}

{(Abgabe am #3)}
\end{center}
}

\begin{document}
\header{5}{18.06.2009}{25.06.2009}{}

\bigskip

{\bf Aufgabe 1  \quad(Bildkomprimierung)}\\
Ziel dieser Aufgabe ist die Komprimierung von Schwarz-Weiss-Bildern.

Laden Sie von der "Ubungsseite das Material f"ur dieses Blatt herunter. Es enth"alt ein {\rm .png}-Schwarz-Weiss-Bild, das Sie komprimieren sollen.

Im GIF-Format reduziert sich die Gr"o{\ss}e von 15012 Bytes des Bildes ($600 \times 200=15000$ Pixel) im PNG-Format auf 11436 Bytes. Schlagen Sie den GIF-Algorithmus. Ihr Algorithmus muss jedoch so gehalten sein, dass er auch auf anderen Schwarz-Weiss-Bildern funktioniert.

Das Material enth"alt zwei Klassen und ein Interface. Das Interface {\rm BWImageReader} stehlt eine die Funktionen {\rm byte getPixel(int i, int j)}, {\rm int getWidth()} und {\rm getHeight()} zur Verf"ugung. Diese erlauben das lesen eines  Schwarz-Weiss-Bildes. {\rm byte getPixel(int i, int j)} gibt Null oder einen Wert gr"o{\ss}er Null zur"uck.

Die Klasse {\rm PNGReader} stellt eine Implementation des Interfaces dar und erlaubt das einlesen von PNG-Bildern.

Die Klasse {\rm BWImageTester} testet zwei Schwarz-Weiss-Bilder auf Gleichheit.

Implementieren zwei Klassen, eine zum Komprimieren des Bildes und eine zum Einlesen des komprierten Bildes. Letztere soll das Interfaces {\rm BWImageReader} implementieren. Benutzen Sie nur Kompressionsalgorithmen, die Sie selbst geschrieben haben, die Benutzung externer Tools ist nicht gestattet.

\end{document}

