\documentclass[a4paper]{article}

\usepackage[german]{babel}
\usepackage{umlaut}
\usepackage{graphicx}
\usepackage{amsfonts}
\usepackage[latin1]{inputenc}

\def\header#1#2#3{\pagestyle{empty}
\noindent
\begin{minipage}[t]{0.6\textwidth}
\begin{flushleft}
\bf \"Ubungen zur Informatik III\\
WSI f\"ur Informatik\\
Lange/Behle/Krebs
\end{flushleft}
\end{minipage}
\begin{minipage}[t]{0.4\textwidth}
\begin{flushright}
\bf Wintersemester 2007/08\\
Universit\"at T\"ubingen\\
#2 %Datum eintragen
\end{flushright}
\end{minipage}

\begin{center}
{\Large\bf Blatt #1}

{(Abgabe am #3)}
\end{center}
}

\begin{document}
\header{2}{30.10.2006}{6.11.2006}


\bigskip




\noindent{\bf Aufgabe 1  \quad 9 Punkte}\\
Geben Sie f"ur folgenden Sprachen m"oglichst kleine Grammatiken "uber\linebreak $\Sigma=\{a,b,c\}$ an:
\begin{enumerate}
\item $\Sigma^*$
\item $\emptyset$
\item $\{\lambda\}$
\item Alle W"orter gerade L"ange
\item Alle W"orter mit einer geraden Anzahl von $a$'s
\item Alle W"orter mit mehr $a$'s als $b$'s
\end{enumerate}

\bigskip

\noindent{\bf Aufgabe 2  \quad 3 Punkte}\\
Beweisen Sie, dass eine Grammatik mit nur einer Produktionsregel nur endliche viele W"orter erkennen kann.
Zeigen Sie, dass bei zwei Produktionsregeln es Grammatiken gibt die unendlich viele W"orter erkennen.

\bigskip

\noindent{\bf Aufgabe 3  \quad 6 Punkte}\\
Ziel dieser Aufgabe ist es, eine Grammatik anzugeben, die kontextfreie Grammatiken beschreibt. Die zu beschreibende Grammatiken sollen dabei folgende Terminal- und Nonterminalmengen benutzen: $T=\{0,1\}$ und $N=\{S,A,B,C,D\}$.

\begin{enumerate}
\item Geben Sie eine Grammatik $G$ an, die alle Produktionen von kontextfreien Grammatiken mit Terminalen $T$ und Nonterminalen $N$ erzeugt. W"ahlen Sie dazu eine geeignete Menge von und Nonterminalen f"ur $G$ und als Terminalmenge die Menge $\{$'0', '1', 'S', 'A', 'B', 'C', 'D', '$\rightarrow$', '\{', '\}', '$\mid$', ','$\}$. Die Grammatik $G$ soll die Produktionen in der Schreibweise $\{S \rightarrow 0S\mid 1A, A \rightarrow AA\mid 0\}$.

Das hei{\ss}t, genau dann ist $G'=(N,T,P,S)$ eine Grammatik wenn $P$ in $G$ ableitbar.
\item Geben Sie einen Ableitungsbaum in Ihrer Grammatik f"ur das Wort an:
$$\{S \rightarrow 0A\mid 0\mid \lambda, A\rightarrow A1A\mid  1 \}$$
\end{enumerate}


\bigskip

\newpage
\header{1B}{30.10.2006}{13.11.2006}

\bigskip


\bigskip

Schreiben Sie einen Intepreter f"ur LOOP Programme, wie in der Definition von Sch"oning (Grundform ohne Erweiterungen).

Ihr Programm erh"alt als Parameter eine Datei und optional weitere ganzzahlige Werte ($<2^{31}$), z.B.:\newline
interpeter loop.prg 2 3 4

Das Programm soll die das LOOP Programm aus der Datei lesen und auf Korrektheit "uberpr"uefen. Dann die Register $x_1,x_2,x_3,\dots$ mit den optionalen Parametern belegen (die "ubrigen Register sind mit 0 initialisiert) und das LOOP Programm ausf"uhren. Am Ende soll Ihr Programm alle Register $x_0,x_1,x_2,x_3,\dots$ ausgeben, die einen Wert ungleich 0 enthalten.

Verwenden Sie zum Programmieren die Programmiersprache C/C++ sowie die Tools yacc/bison, die Ihnen zu einer Grammatik automatisch einen Parser erzeugen. Die offizielle Dokumentation zu bison finden Sie unter:

%http://www.telos.info/Compiler.564+M5d637b1e38d.0.html
%{\tt http://tinyurl.com/tue-info3}
{\tt http://www.gnu.org/software/bison/manual/html\_mono/bison.html}

In der Einleitung wird gezeigt wie ein while Interpreter geschrieben wird. Schreiben Sie bitte selbstst"andig ein Programm, das sich genau an die Syntax aus Sch"oning h"alt.

\bigskip


\end{document}


