\documentclass[a4paper]{article}

\usepackage[german]{babel}
\usepackage{umlaut}
\usepackage{graphicx}
\usepackage{amsfonts}
\usepackage{color}
\usepackage[latin1]{inputenc}
\usepackage[margin=2.5cm]{geometry}
\def\header#1#2#3{\pagestyle{empty}
\noindent
\begin{minipage}[t]{0.6\textwidth}
\begin{flushleft}
\bf \"Ubungen zur Informatik III\\
WSI f\"ur Informatik\\
Lange/Behle/Krebs
\end{flushleft}
\end{minipage}
\begin{minipage}[t]{0.4\textwidth}
\begin{flushright}
\bf Wintersemester 2008/09\\
Universit\"at T\"ubingen\\
#2 %Datum eintragen
\end{flushright}
\end{minipage}

\begin{center}
{\Large\bf Blatt #1}

{(Abgabe am #3)}
\end{center}
\bigskip
}



\begin{document}

\header{9}{08.1.2009}{15.1.2009}

\bigskip
\noindent{\bf Aufgabe 1 \quad 3 Punkte}\\
Zeigen Sie mit Hilfe des Pumping Lemmas f"ur regul"are Sprachen, dass die Sprache\\ $L=\{a^ib^jc^k\mid i=j\mbox{ oder }j=k\}$ nicht regul"ar ist.

\bigskip
\noindent{\bf Aufgabe 2 \quad 4 Punkte}\\
Geben Sie f"ur folgende regul"aren Ausdr"ucke einen "aquivalenten endlichen Automaten an ($\Sigma=\{a,b\}$):
\begin{enumerate}
\item $(aa|aab|aba)^*$
\item $(a(ab|ba)^*b)^*$
\item $\Sigma^*(aab|ab)^*\Sigma^*$.
\end{enumerate}


\bigskip
\noindent{\bf Aufgabe 3 \quad 5 Punkte}\\
Wandeln Sie folgende Automaten in einen DFA um:
\begin{center}
\includegraphics[width=5cm]{aastar}\hspace{3cm}
\includegraphics[width=7cm]{aut1}
\end{center}

\bigskip
\noindent{\bf Aufgabe 4 \quad 5 Punkte}\\
Konstruieren Sie zu folgendem Automaten den Minimalautomaten:
\begin{center}
\includegraphics[width=11cm]{min1}
\end{center}
Geben Sie mit Hilfe des Minimalautomaten einen m"oglichst einfachen NFA an, der dieselbe Sprache erkennt.

\end{document}


