\documentclass[a4paper]{article}

\usepackage[german]{babel}
%\usepackage{umlaut}
\usepackage{graphicx}
\usepackage{amsfonts}
\usepackage[latin1]{inputenc}
\usepackage[margin=2.5cm]{geometry}
\def\header#1#2#3{\pagestyle{empty}
\noindent
\begin{minipage}[t]{0.6\textwidth}
\begin{flushleft}
\bf \"Ubungen zur Informatik III\\
WSI f\"ur Informatik\\
Lange/Behle/Krebs
\end{flushleft}
\end{minipage}
\begin{minipage}[t]{0.4\textwidth}
\begin{flushright}
\bf Wintersemester 2008/09\\
Universit\"at T\"ubingen\\
#2 %Datum eintragen
\end{flushright}
\end{minipage}

\begin{center}
{\Large\bf Blatt #1}

{(Abgabe am #3)}
\end{center}
\bigskip
}



\begin{document}

\header{12}{29.1.2009}{5.2.2009}

\bigskip


\noindent
{\bf Aufgabe 1 \quad(Komplexit"atsklassen)\quad 3 Punkte}\\[1ex]
Unter der Annahme, dass $P\neq NP$, geben Sie eine Sprache $L$ an, die in $NP$ und nicht in $P$ liegt. Begr"unden Sie!

\bigskip

\noindent
{\bf Aufgabe 2  \quad(Independent Set)\quad 4 Punkte}\\[1ex]

\noindent $k$-INDEPENDENT SET

\noindent {\em gegeben:} Ein ungerichteter Graph $G=(V,E)$.

\noindent {\em gefragt:} Gibt es $k$ unverbundene Knoten, d.h. Knoten  $v_1,\dots v_k$ in $V$, so dass f"ur \linebreak alle $v_i,v_j$ ($i\neq j\in\{1,\dots,k\}$) gilt $\{v_i,v_j\}\notin E$.

\bigskip

\noindent Gegeben zwei Zahlen $k_1\leq k_2$, reduzieren Sie $k_1$-INDEPENDENT SET auf $k_2$-INDEPENDENT SET.

\bigskip

\noindent
{\bf Aufgabe 3 \quad(Reduktion)\quad 6 Punkte}\\[1ex]
Reduzieren das Problem $k$-INDEPENDET SET f"ur Graphen mit $|V|$ gerade auf das Problem $|V|/2$-INDEPENDENT SET. Betrachten Sie folgende F"alle:
\begin{enumerate}
\item $k=n/2$.
\item $k<n/2$.
\item $k>n/2$.
\end{enumerate}


\bigskip

\noindent{\bf Aufgabe 4  \quad(Reduktion)\quad 5 Punkte}\\[1ex]

\noindent DOMINATING SET

\noindent {\em gegeben:} Ein ungerichteter Graph $G=(V,E)$ und $k\in\mathbb{N}$.

\noindent {\em gefragt:} Gibt es eine Menge $U\subseteq V$ mit $|U|=k$, so dass f"ur alle
$g\in V\setminus U$ es einen Knoten $v\in U$ gibt mit $\{g,v\}\in E$.

\bigskip

Reduzieren Sie das Problem DOMINATING SET auf KNOTEN�BERDECKUNG und umgekehrt.


\end{document}


