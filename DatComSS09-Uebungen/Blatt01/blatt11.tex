\documentclass[a4paper]{article}

\usepackage[german]{babel}
%\usepackage{umlaut}
\usepackage{graphicx}
\usepackage{amsfonts}
\usepackage[latin1]{inputenc}
\usepackage[margin=2.5cm]{geometry}
\def\header#1#2#3{\pagestyle{empty}
\noindent
\begin{minipage}[t]{0.6\textwidth}
\begin{flushleft}
\bf \"Ubungen zur Informatik III\\
WSI f\"ur Informatik\\
Lange/Behle/Krebs
\end{flushleft}
\end{minipage}
\begin{minipage}[t]{0.4\textwidth}
\begin{flushright}
\bf Wintersemester 2008/09\\
Universit\"at T\"ubingen\\
#2 %Datum eintragen
\end{flushright}
\end{minipage}

\begin{center}
{\Large\bf Blatt #1}

{(Abgabe am #3)}
\end{center}
\bigskip
}



\begin{document}

\header{11}{22.1.2009}{29.1.2009}

\bigskip
\noindent{\bf Aufgabe 1  \quad 6 Punkte}\\
Bestimmen Sie alle Rechtsklassen der folgenden Sprachen:
\begin{enumerate}
\item Alle Bin"arzahlen (ohne f"uhrende Nullen), die durch 5 teilbar sind.
\item Alle Dezimalzahlen (ohne f"uhrende Nullen), die durch 5 teilbar sind.
\item $b(abab)^*$
\item $(ab)^*(ba)^*$
\end{enumerate}

\bigskip

\noindent{\bf Aufgabe 2  \quad 4 Punkte}\\
Zeigen Sie mit dem Satz von Myhill-Nerode, dass die folgenden Sprachen nicht regul"ar sind:
\begin{enumerate}
\item $\{a^nba^n\mid n\in\mathbb N\}$
\item $\{(ab)^n(ba)^n\mid n\in\mathbb N\}$
\end{enumerate}

\bigskip

\noindent{\bf Aufgabe 3  \quad 8 Punkte}\\
Zeigen oder widerlegen Sie, dass die folgenden Sprachen regul"ar sind, dabei ist $\Sigma=\{a,b,\$\}$:
\begin{enumerate}
\item $\{a^*ba^lba^mba^{l+m}\mid l,m\in\mathbb N\}$ 
\item $\{a^kba^lba^mba^{n}\mid k,l,m,n\in\mathbb N, k+m\equiv l+n\pmod 5\}$ 
\item $\{uvw\mid u,v,w\in\Sigma^*, u=v \lor v=w\}$
\item $\{uv\$wx\mid u,v,w,x\in\Sigma^*, u=x\}$
\item $\{ua^{|u|}\mid u\in\Sigma^*\}$
\end{enumerate}
\bigskip

\end{document}


