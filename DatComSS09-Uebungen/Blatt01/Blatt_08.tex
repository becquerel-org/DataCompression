\documentclass[a4paper]{article}

\usepackage[german]{babel}
\usepackage{umlaut}
\usepackage{graphicx}
\usepackage{amsfonts}
\usepackage[latin1]{inputenc}
\usepackage[margin=2.5cm]{geometry}
\def\header#1#2#3{\pagestyle{empty}
\noindent
\begin{minipage}[t]{0.6\textwidth}
\begin{flushleft}
\bf \"Ubungen zur Informatik III\\
WSI f\"ur Informatik\\
Lange/Behle/Krebs
\end{flushleft}
\end{minipage}
\begin{minipage}[t]{0.4\textwidth}
\begin{flushright}
\bf Wintersemester 2008/09\\
Universit\"at T\"ubingen\\
#2 %Datum eintragen
\end{flushright}
\end{minipage}

\begin{center}
{\Large\bf Blatt #1}

{(Abgabe am #3)}
\end{center}
\bigskip
}



\begin{document}

\header{8}{11.12.2008}{18.12.2008}

\bigskip
\noindent{\bf Aufgabe 1 \quad 10 Punkte}\\
Geben Sie f"ur folgende Sprachen jeweils einen endlichen Automaten an. Das Alphabet ist $\Sigma=\{a,b\}$. Geben Sie jeweils den Automaten in Tupelschreibweise an, Sie d"urfen die "Ubergangsfunktion graphisch darstellen.
\begin{enumerate}
\item $L=\emptyset$
\item $L=\{a,b\}^*$
\item $L=\{w\mid w\mbox{ enth"alt genau $3$ $a$'s }\}$
\item $L=\{w\mid w\mbox{ enth"alt mindestens $3$ $a$'s }\}$
\item $L=\{w\mid w\mbox{ enth"alt h"ochstens $3$ $a$'s }\}$
\item $L=\{w\mid w\mbox{ enth"alt eine gerade Anzahl von $a$'s}\}$
\item $L=\{w\mid w\mbox{ enth"alt eine ungerade Anzahl von $a$'s}\}$
\item $L=\{w\mid w\mbox{ enth"alt eine durch 5 teilbare Anzahl von $a$'s}\}$
\item $L=\{w\mid w\mbox{ beginnt mit $aab$}\}$
\item $L=\{w\mid w\mbox{ endet mit $aab$}\}$
\end{enumerate}

\bigskip

\bigskip
\noindent{\bf Aufgabe 2 \quad 3 Punkte}\\
Geben Sie ein Konstruktionsprinzip an, wie Sie zu jeder endlichen Sprache einen endlichen Automaten konstruieren k"onnen.

Wie l"asst sich das Prinzip auf Sprachen erweitern, die Komplemente von endlichen Sprachen sind?

\bigskip
\noindent{\bf Aufgabe 3 \quad 5 Punkte}\\
Geben Sie jeweils einen Automaten an, der die Menge aller nat�rlichen Zahlen in Dezimaldarstellung akzeptiert, die durch 3 teilbar sind/durch 7 teilbar sind.

\end{document}


