\documentclass[a4paper]{article}

\usepackage[german]{babel}
\usepackage{umlaut}
\usepackage{graphicx}
\usepackage{amsfonts}
\usepackage[latin1]{inputenc}
\usepackage[margin=2.5cm]{geometry}
\def\header#1#2#3{\pagestyle{empty}
\noindent
\begin{minipage}[t]{0.6\textwidth}
\begin{flushleft}
\bf \"Ubungen zur Informatik III\\
WSI f\"ur Informatik\\
Lange/Behle/Krebs
\end{flushleft}
\end{minipage}
\begin{minipage}[t]{0.4\textwidth}
\begin{flushright}
\bf Wintersemester 2008/09\\
Universit\"at T\"ubingen\\
#2 %Datum eintragen
\end{flushright}
\end{minipage}

\begin{center}
{\Large\bf Blatt #1}

{(Abgabe am #3)}
\end{center}
\bigskip
}



\begin{document}

\header{6B}{18.12.2008}{8.1.2008}

\bigskip
\begin{center}
\fbox{"Uber die Weihnachtsferien gibt es {\em kein} A-Blatt.}
\end{center}

\bigskip
In der Vorlesung wurden die regul"aren Ausdr"ucke behandelt. Wir definieren hier sternfreie Ausdr"ucke "uber dem Alphabet $\Sigma$.

\begin{itemize}
\item $\emptyset$ ist ein sternfreier Ausdruck.
\item $\lambda$ ist ein sternfreier Ausdruck.
\item F"ur jedes $a\in\Sigma$ ist $a$ ein sternfreier Ausdruck.
\item Wenn $\alpha,\beta$ sternfreie Ausdr"ucke sind, so ist $\alpha\beta$, $(\alpha|\beta)$ sowie $\overline{\alpha}$ ein sternfreier Ausdruck.
\end{itemize}

Dabei bezeichnet $\overline{\alpha}$ das Komplement der Sprache die durch $\alpha$ beschrieben wird, also $L(\overline{\alpha})=\Sigma^*\setminus L(\alpha)$. Die "ubrigen Operationen sind wie bei regul"aren Ausdr"ucken definiert.

\begin{enumerate}
\item Beweisen Sie dass die Sprachen $\Sigma^*$, $a^*$, $(ab)^*$ sternfrei sind.
\item Beweisen Sie, dass die Sprachen die sich durch sternfreie Ausdr"ucke beschreiben lassen, sich auch durch regul"are Ausdr"ucke beschreiben lassen.
\item Programmieren Sie einen Algorithmus der einen sternfreien Ausdruck in einen regul"aren Ausdruck umwandelt.
\item Beweisen Sie dass die Sprache $(aa)^*$ nicht sternfrei ist.
\end{enumerate}

\end{document}


